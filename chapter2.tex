\chapter{Fundamentals of Magnetic Resonance Flow Imaging}
This chapter will explain the basics of MRI and how quantitative flow information can be derived from MR images, during and post reconstruction. The first subsection will introduce how images are formed using MRI physics. The second subsection will then shift focus to flow imaging, presenting the commonly used pulse sequences to image fluid flow in the human body. The quantitative aspects of flow imaging will be explained in detail. This section will also briefly discuss the problems that could arise during quantitative flow imaging, highlighting the need for more accurate flow estimates. The final subsection in this chapter will introduce the application of the aforementioned MRI techniques to blood flow and cerebrospinal fluid flow imaging in the brain. 

\section{Magnetic moment to images}
\subsection{Net magnetic moment }
The classical description of precession using a top analogy is probably not an accurate explanation, but is good for visualization. What actually happens is that there is an energy barrier between the ground and excited states. The difference in the spin population between these states defines the net magnetic moment of an object. We will also talk about the energy equation here. The next important thing here is to bring in the Bloch equations. They essentially describe how the magnetic field changes over time. The concept of transverse and longitudinal relaxation will also be introduced. 
\subsection{Spatial localization}
What good is knowledge of temporal magnetic field variations when what we actually are interested in is the ability to generate an image based on the magnetic field variations. Enter magnetic field gradients that are the foundation of spatial localization. We now are transitioning into the spatial frequency space or k-space as we so fondly are accustomed to calling it. 
\section{Deriving flow information}
Topics to discuss here include : generating velocity maps, angiograms for both quantitative and qualitative information. Here would be a good point to discuss the post-processing stages involved namely : phase difference reconstruction, concomitant field correction and polynomial phase fitting to compensate for eddy current effects. 
\section{Application to blood and CSF flow}